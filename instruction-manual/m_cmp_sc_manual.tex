\documentclass{apl-guide}
\usepackage{subfiles}
\usepackage{lipsum}
\usepackage[utf8]{inputenc}

\title{Harthshorn Coil}
\experiment{SC}
\course{}
\creation{July 21st}
\author{Rhamel Roomes-Delpeache}
\revisions{}
\owner{University of Toronto}
% \labpicture{photos/sample_title_img.JPG}

\begin{document}
\maketitle
\tableofcontents
\section{Introduction}
In this experiement you will be looking at the supercoducting properties of
bismuth-strontium-calcium copper-oxide (\ch{BSCCO}). Superconductivity is a
phenomenom experienced by certain materials, where at certain temperatures,
there is theoretically no electrical resistance. Your job in this lab is to
investigate the validity of the previous statement. During this experiment you
will cover many topic related to experimental physics. You will learn how vacuum
pumps work, take the oppurtunity to connect electronics and perform simple
circuit measurements, calibrate temperatures using trans diode, and work with
liquid nitrogen in a cold metal dewar. 

\section{Background}
\lipsum[1-1]

\section{Experiment}
\lipsum[1-1]

\section{Method}
\lipsum[1-1]

\section{Appendix}
\subfile{cryostat.tex}
\lipsum[1-1]
\end{document}
